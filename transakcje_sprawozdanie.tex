\documentclass[11pt,oneside, a4paper]{article}
\usepackage[utf8]{inputenc}
\usepackage[T1]{fontenc}
\usepackage{polski}
\usepackage[document]{ragged2e}
%packages for row counter in table
%{{{
\usepackage{array,etoolbox}
\preto\tabular{\setcounter{magicrownumbers}{0}}
\newcounter{magicrownumbers}
\newcommand\rownumber{\stepcounter{magicrownumbers}\arabic{magicrownumbers}}
%}}}
%package for code injection
%{{{
\usepackage{listings}
\usepackage{color}
\lstset{
    backgroundcolor=\color{white},
    basicstyle=\footnotesize\sffamily,
    breaklines=true,
    frame=single,
    language=SQL,
    numbers=left,
    numbersep=5pt
}
%}}}
\usepackage{geometry}
    \geometry{
    a4paper,
    top = 2mm,
    left=20mm,
    right = 20mm,
    }

\title{}
\date{}

\begin{document}
\justify
\textbf{Cel badania ->} izolacja:
\textit{READ\_COMMITED}, cecha:
\textit{Niezatwierdzony odczyt} \\
\textbf{Wykonane instrukcje: } \\
Transakcja 1
\begin{lstlisting}
SET TRANSACTION READ WRITE NAME 'Create_Account';
UPDATE ACCOUNTS SET ACCOUNT_BALANCE=3000 WHERE ACCOUNT_NUMBER=1110002;

ROLLBACK;
\end{lstlisting}
Transakcja 2
\begin{lstlisting}

SET TRANSACTION READ ONLY NAME 'Balances';
SELECT ACCOUNT_BALANCE FROM ACCOUNTS WHERE ACCOUNT_NUMBER=1110002;
\end{lstlisting}
\textbf{Otrzymane wyniki:} \\
    Przed rozpoczęciem testów pole o id=1110002 miało wartość 20 000. Transakcja 2, zwróciła  wartość 20 000.\\
\textbf{Wnioski} \\
    Baza danych przy domyślnym poziomie izolacji: READ\_COMMITED jest odporna na niezatwierdzony odczyt. \\
\newpage

\textbf{Cel badania ->} izolacja:
\textit{READ\_COMMITED}, cecha:
\textit{Niepowtarzalny odczyt} \\
\textbf{Wykonane instrukcje: } \\
Transakcja 1
\begin{lstlisting}
SET TRANSACTION READ WRITE NAME 'Update_Account';
SELECT ACCOUNT_BALANCE FROM ACCOUNTS WHERE ACCOUNT_NUMBER=1110002;
UPDATE ACCOUNTS SET ACCOUNT_BALANCE=3000 WHERE ACCOUNT_NUMBER=1110002;
COMMIT;
SELECT ACCOUNT_BALANCE FROM ACCOUNTS WHERE ACCOUNT_NUMBER=1110002;
\end{lstlisting}
Transakcja 2
\begin{lstlisting}
SET TRANSACTION READ WRITE NAME 'Read_Balace';
SELECT ACCOUNT_BALANCE FROM ACCOUNTS WHERE ACCOUNT_NUMBER=1110002;

SELECT ACCOUNT_BALANCE FROM ACCOUNTS WHERE ACCOUNT_NUMBER=1110002;
\end{lstlisting}
\textbf{Otrzymane wyniki:} \\
    Przed rozpoczęciem testów pole o id=1110002 miało wartość 20 000. Transakcja 1 oraz 2, zwróciła  wartość 20 000. Transakcja 1 dokonała updatu rekordu id=1110002. Po zatwierdzeniu zmian obie Transakcje zwróciły te same wartości 3000\\
\textbf{Wnioski } Baza przy poziomie izolacji Read\_Commited nie jest odporna na niepowtarzalny odczyt.\\
\newpage
\textbf{Cel badania ->} izolacja:
\textit{READ\_COMMITED}, cecha:
\textit{Fantomy} \\
\textbf{Wykonane instrukcje: } \\
Transakcja 1
\begin{lstlisting}
SET TRANSACTION READ WRITE NAME 'Update_Account';
SELECT SUM(ACCOUNT_BALANCE) FROM ACCOUNTS;


UPDATE ACCOUNT SET ACCOUNT_BALANCE = ACCOUNT_BALANCE*1.5;
COMMIT;
\end{lstlisting}
Transakcja 2
\begin{lstlisting}
SET TRANSACTION READ WRITE NAME 'Update_Account';
SELECT SUM(ACCOUNT_BALANCE) FROM ACCOUNTS;
INSERT INTO ACCOUNTS (ACCOUNT_NUMBER, ACCOUNT_BALANCE) VALUES (SEQ_ACCOUNT_NUMBER.NEXTVAL, 10000);
COMMIT;
\end{lstlisting}
\textbf{Otrzymane wyniki:} \\
    Zapytanie SELECT Transakcji 1 oraz 2, zwróciło  wartość 33 000. Transakcja 2 wprowadziła nowy rekord do bazy danych. Po commitcie Transakcji 2, Transakcja 1 dokonała updatu wartości kolumny ACCOUNT\_BALANCE powiększając wszystkie wartości o 50\%. Łącznie z nowo dodanym.
\textbf{Wnioski } Baza przy poziomie izolacji Read\_Commited nie jest odporna na fantomy.\\
\newpage
%-----------------------------------------------------------
% IZOLACJA SERIALIZABLE
\textbf{Cel badania ->} izolacja:
\textit{SERIALIZABLE}, cecha:
\textit{Niezatwierdzony odczyt} \\
\textbf{Wykonane instrukcje: } \\
Transakcja 1
\begin{lstlisting}
SET TRANSACTION READ WRITE NAME 'Create_Account';
UPDATE ACCOUNTS SET ACCOUNT_BALANCE=3000 WHERE ACCOUNT_NUMBER=1110002;

ROLLBACK;
\end{lstlisting}
Transakcja 2
\begin{lstlisting}

SET TRANSACTION READ ONLY NAME 'Balances';
SELECT ACCOUNT_BALANCE FROM ACCOUNTS WHERE ACCOUNT_NUMBER=1110002;
\end{lstlisting}
\textbf{Otrzymane wyniki:} \\
    Przed rozpoczęciem testów pole o id=1110002 miało wartość 4500. Transakcja 2, zwróciła  wartość 4500.\\
\textbf{Wnioski} \\
    Baza danych przy izolacji: SERIALIZABLE jest odporna na niezatwierdzony odczyt. \\
\newpage

\textbf{Cel badania ->} izolacja:
\textit{SERIALIZABLE}, cecha:
\textit{Niepowtarzalny odczyt} \\
\textbf{Wykonane instrukcje: } \\
Transakcja 1
\begin{lstlisting}
SET TRANSACTION READ WRITE NAME 'Update_Account';
SELECT ACCOUNT_BALANCE FROM ACCOUNTS WHERE ACCOUNT_NUMBER=1110002;
UPDATE ACCOUNTS SET ACCOUNT_BALANCE=3000 WHERE ACCOUNT_NUMBER=1110002;
COMMIT;
SELECT ACCOUNT_BALANCE FROM ACCOUNTS WHERE ACCOUNT_NUMBER=1110002;
\end{lstlisting}
Transakcja 2
\begin{lstlisting}
SET TRANSACTION READ WRITE NAME 'Read_Balace';
SELECT ACCOUNT_BALANCE FROM ACCOUNTS WHERE ACCOUNT_NUMBER=1110002;

SELECT ACCOUNT_BALANCE FROM ACCOUNTS WHERE ACCOUNT_NUMBER=1110002;
\end{lstlisting}
\textbf{Otrzymane wyniki:} \\
    Przed rozpoczęciem testów pole o id=1110002 miało wartość 4500. Transakcja 1 oraz 2, zwróciła  wartość 4500. Transakcja 1 dokonała updatu rekordu id=1110002. Po zatwierdzeniu zmian Transakcja 2 zwróciła wartość 4500, natomiast Transakcja 2 zwróciła wartośc 3000\\
\textbf{Wnioski } Baza przy poziomie izolacji SERIALIZABLE jest odporna na niepowtarzalny odczyt. Update transakcji nr 1 nie został zacommitowany, więc transakcja 2 zwróciła 2 razy tę samą wartość.\\
\newpage

\textbf{Cel badania ->} izolacja:
\textit{SERIALIZABLE}, cecha:
\textit{Fantomy} \\
\textbf{Wykonane instrukcje: } \\
Transakcja 1
\begin{lstlisting}
SET TRANSACTION READ WRITE NAME 'Update_Account';
SELECT SUM(ACCOUNT_BALANCE) FROM ACCOUNTS;


UPDATE ACCOUNT SET ACCOUNT_BALANCE = ACCOUNT_BALANCE*1.5;
COMMIT;
\end{lstlisting}
Transakcja 2
\begin{lstlisting}
SET TRANSACTION READ WRITE NAME 'Update_Account';
SELECT SUM(ACCOUNT_BALANCE) FROM ACCOUNTS;
INSERT INTO ACCOUNTS (ACCOUNT_NUMBER, ACCOUNT_BALANCE) VALUES (SEQ_ACCOUNT_NUMBER.NEXTVAL, 10000);
COMMIT;
\end{lstlisting}
\textbf{Otrzymane wyniki:} \\
    Zapytanie SELECT Transakcji 1 oraz 2, zwróciło  wartość 63 000. Transakcja 2 wprowadziła nowy rekord do bazy danych. Po commitcie Transakcji 2, Transakcja 1 chciała dokonać updatu wartości kolumny ACCOUNT\_BALANCE powiększając wszystkie wartości o 50\%. Baza danych zwróciła error blokując UPDATE.
\textbf{Wnioski } Baza danych jest odporna na fantomowy odczyt.
\newpage

\textbf{Cel badania ->} izolacja:
\textit{READ\_COMMITED + FOR UPDATE}, cecha:
\textit{Niezatwierdzony odczyt} \\
\textbf{Wykonane instrukcje: } \\
Transakcja 1
\begin{lstlisting}
SET TRANSACTION READ WRITE NAME 'trans1';
SELECT ACCOUNT_BALANCE FROM ACCOUNTS WHERE ACCOUNT_NUMBER=1110002 FOR UPDATE;
UPDATE ACCOUNTS SET ACCOUNT_BALANCE=9000 WHERE ACCOUNT_NUMBER=1110002;
COMMIT;
\end{lstlisting}
Transakcja 2
\begin{lstlisting}
SET TRANSACTION READ WRITE NAME 'trans2';
SELECT ACCOUNT_BALANCE FROM ACCOUNTS WHERE ACCOUNT_NUMBER=1110002 FOR UPDATE;

UPDATE ACCOUNTS SET ACCOUNT_BALANCE=5000 WHERE ACCOUNT_NUMBER=1110002;
COMMIT;
SELECT ACCOUNT_BALANCE FROM ACCOUNTS WHERE ACCOUNT_NUMBER=1110002;
\end{lstlisting}
\textbf{Otrzymane wyniki:} \\
    Transakcja 2 czekała na zakończenie transakcji 1. Transakcja 1 pierwszy odczyt danej był 3000, pierwszy odczyt transakcji 2 był 9000, kolejny SELECT zwrócił wartośc 5000.
\textbf{Wnioski } Baza danych przy dodaniu do zapytania FOR UPDATE zachowuje się podobnie jak przy poziomie izolacji SERIALIZABLE, gdzie jedna transakcja czeka na wykonanie się poprzedzającej transakcji.
\end{document}
